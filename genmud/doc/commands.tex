\chapter{Commands}


\section{Introduction}


When someone connects to the game, they interact with it by typing
things in. The commands come into an interpreter and get processed to
make the game do stuff. Other things besides players can issue
commands, but things don't always work the same between players and
non players.


\section{The cmd struct}

The struct used to let things have commands is called struct cmd:


struct cmd

{

  char *name;                        This is the "name" of the command
				       you type in to use it. 

  COMMAND\_FUNCTION *called\_function;  This is the actual pointer to the
					actual function which is executed
					when you do this thing. 

  int level;                          Min Level to Use 

  CMD *next;                          Next command in the list 

  bool log;                           Is this always logged? 

};


The name is the string of letters that a character must type in to use
that command. The command parser attempts to find commands even on
partial strings of letters. The commands are hashed using only the
first letter of the command, so if you wish to use a single letter
command, the game will look for the commands that were added last to
the lists (so they are at the front of the lists) when it looks
something up.

The called\_function is the actual piece of code to be called when the
command is executed. This is just a pointer to the function in
question, and when it gets called, 

(*cmd->called\_function)(th, arg) 


is called to execute the code.

The level is the minimum level required to use this command. Useful
for highlevel commands.

Log determines if this command is always logged or not. This is useful
for messy commands like delete and freeze where you want to track the
stuff your staff is doing to your players. Commands can also be logged
by setting a log bit on a player, like log bob, or by typing log all
which will log every single command by every player.


\section{The Interpreter}

The interpreter works by passing two arguments to interpret(). The
first argument is the thing attempting to do a command, and the second
is an argument containing the command and any other arguments needed
for the command.


The game first does some checking to make sure that the thing and
argument exist. Then, it checks for in-game restrictions on commands
due to paralysis or freezing or other things, and it removes hidden
from the thing doing the action.


The next step is to see if the command is a channel (since those are
stored on a separate list and accessed through a common interface.
Then, clan editing commands are checked, and then the commands
themselves are checked by matching the word to the command name.
If none of these ended up with a match, then the command is checked
against socials. Then, the actual command is called if there was one.


Finally, everything in and around the thing doing the command gets
checked for command scripts (verb binding), and the logs are updated,
if necessary.





















