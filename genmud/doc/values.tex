\chapter{Values}

\section{HOWTO}

If you want to add a value, first find the list of values in 
serv.h Look for the string ``VAL\_MAX'', and add your new value
to the list using the VAL\_MAX number as the new value number.
Increment VAL\_MAX. If you remove a value, leave a placeholder
or you will mess up all of the numbers that come after that
number. 

The next step is to look in const.c for value\_list. Add a new
row of information for your value down at the end of the list,
just before the final entry for VAL\_MAX. Don't mess up the order
of the values or bad things will happen. This setup is intentionally
delicate for the sake of efficiency.

For completeness, you should really add an entry to the helpfiles
and this chapter for the sake of the documentation. I forget to
do this sometimes, but I know that I really should do it.

\section{Why?}

Ok, you might be wondering why I made these little data chunks that I
use for everything. Ok, I admit it. It comes down to laziness. Look,
data comes in number form or string form. I didn't feel like having to
code a little dorky piece of data for every little thing I wanted to
make. I didn't want to have to store files as binary in order to save
me trouble. I didn't want to have to deal with allocating and
destroying random bits of data 10000 times a second just because
everything has its own special little data ``needs''. I also didn't
want to make a general database/scriptengine that would force me to do
all my coding in some crappy language I would have to make up, and which
wouldn't work a tenth as good as the standard tools.

So, instead of going to the various extremes of having pointers to
``mob data'' or ``room data'' or ``obj data'', or having a void
*pointer and using some kind of fake (or real) inheritance to try to
keep track of it all, or having some sort of totally general database
where I still have to keep track of tons of things, I decided on this.

\section{What is it}

Every thing shares some common types of data. Other than that, they
all have their own quirks and data types. So, I decided to make an
interchangeable data chunk that I could just plop in and out of
everything. I suppose there are other ways of doing this like byte by
byte data, or special databases or void * pointers to chunks of
different sizes, but I don't care if this is slick. I want it to be
simple and easy to work with and to not cause me too much misery.

Each value consists of a tag or type, a pointer to a value, 6 ints and
a char *. Values represent lots of different things, but here are a
few examples: exits, weapons, money, guarding, casting.

If you want to add values to the game, you can do this by first making
a new VAL\_SOMETHING constant in serv.h. Then, goto const.c and add a
row to the values\_data table so that you can access the value within
the editor. Then, add code to access that value. You use the
find\_next\_value function, or the FNV macro o access a value within
the game.

If you ever want to remove a value, don't just remove it from the list
in const.c Keep a ``blank'' value in there someplace so you don't
screw up the list. I think it's set up so that the place in the array
determines the value number set on an item, so you have to be sure to
keep the blanks filled in.


\section{Editing}

Values are edited in two main ways. First, in the editor, you can type
any or all of the word value, like v, va, val etc..., a value type
(explained below), and then v0-v5 or word, and then a piece of data.
The reason things stop at v5 is that NUM\_VALS = 6. That means the 
array index goes from 0 to 5.

For example, typing val weapon v0 5 would set the 0th value of a
weapon value on the item you are editing to 5. It would also create
the value if needed. It is also possible to delete values by typing
val weapon delete, for example. Each item can only have one of each
type of value. I considered allowing more, but it seemed like too much
of a hassle, and this has worked well enough so far.

For exits, there are also other things you can do to edit
values. Since the main place values are used is for exits, it would be
too much of a hassle to constantly type val eexit v0 4344 and stuff
like that, so there are shortcuts which are explained
here. Essentially, there is a shortcut called dig you can use when
editng rooms. You should be inside of the room you are editing in
order to use dig.

The way this works is that you type $<$direction$>$ dig $<$blank/number$>$.

The direction can be a single letter like e, or the whole word like
east. If you dig and don't specify a number, the dig command will look
in the area you are editing for the next available room thing that
hasn't been created. It will then create that room thing, and it will
make a value of the proper type with the v0 set to the new room
number. You will then be transported to the new room you just created,
and an exit in the new room will be placed which heads back to the
original room you were in.

If you type a specific number, then the game will automatically make
the link to the room, it will make a link from the room back to your
current room, but it will not move you. This difference in sometimes
moving vs not moving is just because this is how it's easiest for
me. If I want to make an NxK block of rooms, I can make a row of N
rooms, and then another row parallel to the first, and when it comes
time to link them up in the opposite direction, it just seemed easier
to keep you in the original room.


Now, here's a list of the values and what each of the numbers in each
represent. This is all hardcoded, and each value can only hold a
limited amount of data, so be careful when you want to expand
this. Btw, if you ever need more numbers per value, my suggestion is
to do things in 2 steps: load up the game using the old number of ints
per value, and resave every area where the new ints needed are all
0's. This is safer than trying to muck around with getting errors in
each value when they are loaded in. So, for example if you need to go
from 6 to 7 ints per value, or you do something else, then make the
write\_value() function use the new number, and the read\_value()
function still use the old amount, then once the data is saved, change
read\_value to the new amount. This is a safer way of changing the
savefiles than trying to do it all at once, although it should work
unless you have a number for a word...then you are in deep doodoo.

I will let v0-v6 represent val->val[0] to val->val[6] for shorthand notation.

\bul eexit wexit sexit nexit uexit dexit -- These are all exit
values. In each case, v0 = the room number you go to, v1 = current
door flags, v2 = reset door flags, v3 = strength. This part is odd and
a hack. :P The formula for this number is the number is equal to
1000*max\_strength + curr\_strength. That means all the 50050 things
around the game mean the doors have 50 max str, and 50 current str. v4
is a trap number (based on trap spell number), and v5 is the key
number that opens the door. The reason I do this is so that you can
have several doors in one room each having different keys to open
them. You can also have one key opening several doors. There are
probably better or different ways to handle keys, but this works for
now.

\bul iexit -- This is a portal. v0 tells what room the portal goes
to. You use portals by typing enter itemname. If an item doesn't have
a portal in it, and you enter it, you end up inside of that object
(like a boat). I haven't decided if I want to make portals do nasty
things to people like damaging them or stripping them of their eq. One
neat little tidbit is that you can look into a portal and see the room
where it leads. This is useful for portalwhores who don't want to keep
going out to see if enemies have left their hometown.

\bul oexit -- This is a lid on a chest or container. My analogies were
breaking down at this point, but essentially it works in the same way
that doors do where you may need a key to open it, and it can be
closed and locked and whatnot.

\bul nouter, e, w, s, u , d, i , oouter. These are all clmbable
objects. v0 = the room number it leads to. v1-v2 = flags. That means
that one item can be climbed in several directions. There might also
be flags someday which do things to people when they climb in that
direction, but I haven't done anything with it yet.

\bul weapon -- These are weapons. The damage range is from v0 to v1
points. NOT DICE!!!! v2 is the type of weapon: 0 = slashing, 1 =
piercing, 2 = concussion, 3 = whipping. So far, only piercing really
does anything (for backstabs). You can also set weapons to have
different numbers of attacks, like high damage weapons might be slow
(- attacks), and light weapons might have more attacks. Be very
careful of changing this. v4 and v5 deal with poisons. You will be
able to coat weapons with poisons, and what happens essentially is
that you get v4 shots of poison number v5. You can set up weapons to
have poisons from the time they are created, but it isn't advised.

\bul powershield -- These are devices that absorb damage and
regenerate quickly. v1 = current power, v2 = max power.

\bul gem -- These things store magical power so you can cast more spells.
The detailed description will come in the spells section, but for
right now, v0 = mana types, 1 = air, 2 = earth, 4 = fire, 8 = water,
16 = spirit. v1 = current mana, v2 = max mana, v3 = level (added to
your natural level, NOT the new level you can use to cast!!! Be
careful about this!)

\bul armor -- This is protective equipment. v0 = head armor, v1 = body
armor, v2 = arms armor, v3 = legs armor. Each armor point deducts
roughly half a damage point every time the wearer is hit in that
location.

\bul food -- You eat this. v0 = amt of hours of ``fullness'' the food
gives, v2 = flags...maybe later. v3-v5 = spells cast on eater. Pills
== food here.

\bul drink -- You drink this. v0 = curr\_drinks, v1 = max\_drinks (0 =
unlimited drinks), v3-v5 = spells it casts on the drinker (potions =
drinks here, so potions should generally have one drink in them!).

\bul magic\_item This doesn't really do anything yet, but the idea
will be to make scrolls, rods, staves, wands, etc... using this
thing. v0 = flags (Area...etc), v1 = curr\_charges, v2 = max\_charges,
v3-v5 = spells cast.

\bul tool -- Don't know what this will do, but v0 will be the type. Eg.. 
pickaxe, lockpicks?  For now, v2 = 1 is mountain boots.

\bul raw -- These are raw materials for building things. Probably
won't use these as much as I had planned, since mining turns out to be
pretty cheater. v0 = type..mineral, wood etc..., v1 = rank, v2 = number maybe??

\bul ammo -- these are things that get shot from ranged weapons. v0-v1
= min-maxdam, like weapons. v2 = curr\_shots, v3 =
max\_shots..possibly for reloading clips...not really used atm,
though. v4 = ammo\_type (must match ranged weapon v4).

\bul ranged -- these are weapons that fire ammo at people. v0 = speed
(how long you have to aim this. v1 = range, how far away you can
shoot, v2 = number of shots fired at once, so you can make machine
guns. :P v3 = multiplier, like in angband...the bows and stuff
multiply the damage done. So, you might have to choose between more
shots, or more powerful shots. v4 = ammo type (must match the v4 of
the ammo loaded into it).

\bul shop -- These are things that let mobs run shops. The v0 and v1
are the opening and closing hours respectively. If they are the same,
the shop is open 26 hours a day. If the first is after the second,
then the shop stays open from the first hour overnight to the second,
otherwise the shop stays open from hour v0 to hour v1. The v2 is the
shop profit percent, basically how much it pays relative to the
``true'' cost of an item. A 100 here is normal. Shops can also make
things...like baking bread or something. If the hour is a multiple of
v3, it will make an item number v4. It stops when it has 10 in the
inventory I think. If you try to sell too many of an item at the shop,
it will pay less for something the more of it it has. The v5 tells how
many items the shop must have before it starts paying less for a
certain item. The word for the value is a list of item types the shop
works with. Basically, if you want a weapon shop, type word weapon, if
you want something that sells food and drink, type word food
drink. Then, the shop will sell items like that. Shops repair
everything they sell so there's no need to make special repair flags
or anything like that.

\bul teacher teacher1 teacher2 -- v0-v5 are the spell numbers that
things teach. This is one of the places where I hadn't made values a
fixed length since I needed to make several teacher values to get
around the problem. If this value is in a room, you can have a
teaching room without a teacher. If it's on a mob, you can have a mob
teacher. If it's on an item, it acts like a book in the sense of
setting you to the minimal amount to improve on your own, and then the
book disappears.

\bul money Money is a value. It's easier than having 5-6 ints on each
thing. Starting with v0, the amounts are copper, silver, gold,
platinum, obsidian, adamantium. Each subsequent amount is the previous
value times 10. If this is set on a thing that gets reset, the amount
is randomized a little for each instance.

\bul guard -- This is a thing that guards a direction or a thing. If a
room guards, it blocks things moving out of it. A mob or object can
block something in the room it's in. V0 = flags, 1 = guard\_door, 2 =
guard\_door\_align, 4 = guard\_door\_pc, 8 = guard\_door\_npc, 16 =
guard\_doc\_society, 32 = guard\_door\_vs\_above (used for guarding vs
levels. v1-v3 = things it guards. If the thing being guarded is an
object, if the object is picked up, the picker-upper gets attacked. If
the thing is a mob, the guarder attacks anyone attacking the
guarder. v4-v5 = clan/sect guards...guard vs all except those in the
correct clan/sect. word = string of direction letters from newsud
telling what directions it guards.

But this gets more complicated. There are up to 10 kinds of clans or
sects. If v4 is from 0 to CLAN\_MAX, this means we are guarding for a
certain kind of clan.. 0 = clan, 1 = sect for now. v5 is the number
we're guarding for. So, if you re clan[v4] != v5, you cannot pass the
guard.

On the other hand, if v4 > CLAN\_MAX, then v4 tells what we guard
against in terms of player stats or levels. What happens is this. if
type == 10, we guard vs level, 11 = guard vs remorts, 12 = guard vs
mana, 13 = guard vs max\_mana, 14 = guard vs hps, 15 = guard vs max
hps, 16 = guard vs bank, 40 to 40 + stat\_max - 1 = guard vs certain
stats, 60 to 60 + guild\_max = guard vs guild levels. In each case,
you cannot pass if your stat is too low. v4 sets the minimum amount
you need to pass. I.e. you could set v4 = 41 (intelligence) and v5 =
20 and you would need at least a 20 int to pass. There is a flag
GUARD\_VS\_ABOVE which is useful if you want to say make a newbie
area. You can make a blocker that blocks vs level 25+, and another
blocks which blocks vs 1 or more remorts.

\bul cast -- these are things that cast spells at intervals. If a room
casts, it casts on things in it. These are room traps, and rooms that
cast helpful spells. This sort of casting CAN be based on alignment if
you want. If the room align is 0, it treats all other aligns the same,
and if the align is nonzero, it helps its own align and hurts
others. Mobs can cast on people with helpful spells if they are of the
proper align, and harmful spells against people they fight. Objects
will cast spells on behalf of their owner..i.e. they will heal the
owner periodically if they cast that, and they will cast spells on
their owner's enemy.


\bul map -- This is a map item. It uses the mapping code to draw an ascii map of the area surrounding the location specified by v0 if the player looks at it.
I may add something someday where maps can only be used a few times
before they go poof, but it isn't in yet.

\bul guild -- these are for guildmasters. v0 = guild type (shifted by
1 so v0 = 0 is nothing). v1 = min tier, v2 = max tier. This means you
can have lowlevel guildmasters out and about and make the higher tier
ones really hard to get to.

\bul feed -- This is used for some kind of ecology I started working
on. The idea is the mobs have a list of things they like to eat. Every
few minutes, they scan out a certain number of rooms and if they find
something they like to eat, they go and hunt it down and try to kill
it. If not, they stay hungry and they can eventually get weak and
die. v0 = range they will hunt, v1 = internal, number of times they
failed to find anything, v2-v5 = vnums of things they like to eat.

\bul shop2 -- these are for player run shops. v0 = cost to take over
v1 = percent you can skim off the top at any time, word = the name of
the owner. This code isn't so great right now since if a player-run
shop mob is killed, it dosen't come back.

\bul light -- v0 = hours left, v1 = max\_hours, v2 = flags (is it
lit). Used to light up rooms and stuff.

\bul randpop -- these are used to set up random resets. They will be explained in the section on resets.

\bul pet -- These are pets, ``charmed'' creatures and also things you summon
        that follow you. They can either be set to be timed, or are permanent.
So, that's all the values for now. If you find you need new data to add to things, and it won't work if it's a flag...make a new value. They are here to make your life easier, and don't worry about using a little extra memory...IMO it's more important to have an easier time expanding the code, since most things will only have 1-2 values at the most anyway.

\bul society -- These creatures are in a society. v0 = society number. v1 =
caste number. v2 = caste flags. v3 = caste activity or characterization. For
example, it's the kind of raw material a worker looks for, and it's the kind
of warrior something in the warrior caste is. v5 = the society that captures
this one (society members can take each other prisoner when they raid and
then players can free the prisoners as quests...) v6 = the room where this
thing goes to sleep at night. Society creatures go to sleep and wake up
in a cycle that can be set to be daytime sleeping if the society has
a SOCIETY\_NOCTURNAL flag or no sleeping at all if the society has a 
SOCIETY\_NOSLEEP flag.


\bul build -- These chunks of data represent how built up a room owned
by a society is. It makes the room look different like A Hobgoblin Village
instead of A Grassy Field. These values also give defenders a combat
bonus. v0 = society number v1 = build tier v2 = build rank (within a tier
so that builders have to "build" several times to improve the city tier)
v3 = what caste building is here (each type of CASTE\_FLAG has a certain
building type that goes along with it (Eg. Pavilion goes with leaders,
academy goes with mages...). v4 = how burnt the room is v6 = how many hours 
of "decay" this room has had. When societies die, their city rooms fall
into disrepair and eventually get overgrown and vanish. This keeps track
of how long the city has spent at each level of disrepair.

\bul popul -- These are used in the wilderness. The wilderness map is
divided up into sectors that contain populations based on societies
that fight and expand over time. When you walk through the
wilderness, your encounters are not totally random. They can be groups
of animals, or you may encounter members of a population that live
in that sector. The only part of this that's used is v0 = the society
number that this mob is based upon. This will be explained more later on.
See WILDERNESS\_README.DAT in the /wld directory to see how to set up
the wilderness. It should only consist if $\#$defining something and
altering the makefile slightly.

\bul madeof -- This is used for making items in terms of other items.
v0 = process needed to make this item. v1-v6 = items needed to make this
item. For example, to make bread, you may need to bake water, flour and
sugar. This isn't implemented yet.

\bul replaceby -- This is used so that if a timed object has this
value, it gets replaced by something else instead of just disappearing.
The syntax is the same as for VAL\_RANDPOP so you can have very complex
replacement schemes.








