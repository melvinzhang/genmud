\chapter{AI}

\section{Introduction}

One of the neat aspects of this code is the AI and societies
code. Basically, there's a simple RTS inside of the MUD. There are
groups of mobs that go about their lives collecting resouces,
fighting, building cities and settling new lands all without player
interaction. What this means is that players are embroiled within a
war and they won't be able to predict what the game will look like
from day to day. 

For example, a society of trolls may set up shop on
the main highway leading out of your hometown. If they do it at night
and noone stops them, they will set up guard posts which could block
people into their city. Then, they will have to be exterminated over
time, since even if they're all killed off, they may keep coming
back. Eventually they can be destroyed, and not only are their enemies
out there to interact with, but players can also help other societies
by giving them resources and helping them when they're being
attacked. 

I also believe that this kind of simulation is the future of games.
There's no way to generate enough content by hand, so you may as well
just start working on making better simulations to do it for you.
It won't be as good as hand-generated MUDs for a long time, but
eventually it will be.



\section{Tactical AI}

The combat AI in the game comes in two parts; the initiating combat
AI, and the combat actions AI.

\subsection{Initiating Combat}

Things in the game initiate combat for many reasons. They may have
been attacked by something previously, they may have a script that
keeps spamming ``kill foo'', they may have a VAL\_FEED on them and
they're looking for some prey, or they may be attacking enemies of
their society or alignment. The way to determine if something is an
enemy or friend is to use the is\_enemy() and is\_friend() functions
in fight.c.

These attacks only get checked during the thing\_update() for most
things, using the function attack\_stuff(). This checks the room the
thing is in for anything that is an enemy, and if it finds one, it
attacks. attack\_stuff() also gets called during the fast\_update,
which means that players will see things attacking more frequently
than mob vs. mob attacks will occur. 

Society mobs also do other things. Since they go on raids and they
scan for enemies near their locations (if they're combat society
members), they also initiate combat by hunting things that are near
them. When they find their victims, then they attack the victim. 

Mobs can also find things to attack by responding to yells either from
attackers warning that there are enemies nearby, or from victims
yelling that they've been attacked. That means that you can't
guarantee a quiet raid just by letting all of the mobs attack
you. They may yell when they attack you, and help may come for them.

Finally, mobs may attack if they're in the room with some combat and
they need to assist their allies. This includes assisting players,
pets assisting things, and mobs of the same vnum and society (if any)
and alignment helping each other. Note that this doesn't mean that
mobs of the same number always assist each other, since it is possible
to have two different societies with the same mobs in different
alignments, which would make them enemies. Therefore, you should be
very careful about using vnum to determine who's friends with whom.

\subsection{Combat AI}

The AI that the mobs use in combat is hardcoded. I tried to do some
simulations of large numbers of combats to get an idea what the most
effective tactics worked, but it wasn't working since there were too
many variables and I wasn't doing it right. So, I just coded what my
tactics would be and I hope that's good enough.

Basically, combats in the MUD will be 1 on many or 1 on 1 sometimes,
but I expect many on many battles over time, simply because the
societies use numbers to function well.

Each round in combat, a fighting mob will have chance to use some AI.
When it does this, it calls combat\_ai() in combatai.c and that
function does the following:

First, it counts up the number of friends and enemies in the room
relative to this mob and also their powers and total hps. These three
numbers are used to tell roughly how the battle is going. They are
used to calculate a number called relative\_power which is a number
from 0 to 24. 

If the number is near 0, the mob is losing the battle,
and may try to flee. Otherwise, if the relative power is fairly high
and the mob is not too hurt, it may try to rescue an ally. If the
battle is tilted a lot in its favor, it may try to attack the
strongest pc opponent. Otherwise, it will generally go after the
weakest opponent.


Assuming that none of those rescues or switching attacks take place,
the mob will then decide to bash or tackle.  These are ``restraining''
skills since they make it harder for the victim to flee. They are
generally only used if the mob is winning the battle, and they are
used much more frequently if the victim is really hurt,since they
don't do damage and they make the attacker have a harder time fleeing,
too. That means that mobs should avoid bashing and tackling much,
until they're fighting something that gets really weakened, then they
should go after it and try to keep it from fleeing hurt.

If they haven't done anything else to this point, the mobs will
generally kick, and sometimes flurry (since kick is more consistent
damage and flurry can be either really cheater or really bad depending
on the defenses of the victim).

One ting that I may do at some point is give mobs a "bravery" stat that
will modify the relative\_power number up or down for that individual
mob so that they don't all do exactly the same thing. (Eric Young idea)



\section{Pathfinding}

The pathfinding in the game is done with a breadth-first search, where
certain rooms are culled as being impassible. The reason for this is
that there is no notion of direction in the world, so I can't use
something like A* to steer the search in the right direction. One
thing I may do at some point is have a hierarchical system where the
connections between areas are checked first, then paths through each
area are calculated. I have avoided this so far since the timing
hasn't been a problem yet. Also, since if many things are going to the
same place, they all follow the same trail generated by the first
thing to look for the path. And beyond that, most of the paths needed
are really close by from the start point, so the hierarchy wouldn't
get used that much anyway.


Pathfinding starts when a thing is hunting something. It does a
breath-first search, where each room is checked based on what
type of thing is being hunted.


I had a question about what
\vv
"Tracks of An Open Field leading out to the north, in from the nowhere, with 4
hours left." 
\vv
means when you edit a room.

What happens is that when mobs and players move around they leave tracks
(sometimes). When mobs pathfind, they do a search of the room graph and when
they find the room where they're going, the game calls place\_backtracks()
that leaves a trail of tracks from the mob's room to the ending room. The
thing that these tracks point to is the room itself, hence the thing like
"Tracks of An Open Field" and the in from nowhere is because these tracks will
never be seen by players. They are only used internally by mobs so I don't
need to add the dir that the thing came from (saves a little cpu).

The reason I do this is again so that if a lot of things are going from one
location to another (like raiding or settling) then the first one calculates
a path, then the rest of them search for the tracks that were laid down
by the first thing and follow the tracks room by room. It's much more
efficient.


\begin{enumerate}

\item If the thing being hunted is a room, the room is checked to see
if it's that room. Otherwise, it checks the tracks in that room to
see if one of them points to the proper room. Things hunt for
rooms by hunting a number rather than a name.

\item If it's something else, the contents of the room are checked.
The way it works is that if you don't have a hunt\_victim
pointer set to something, then you're searching for a name.
It searches the contents of the room for something with that
name, and then searches the tracks in the room to see if something 
with that name left a track here recently.


\item If the thing is not found, and a trail is not found, then the
adjacent rooms are checked. Of course, rooms are only checked once,
and the room being checked cannot be impassable to the thing doing the
hunting. If these BADROOM\_BITS are set and the thing can't pass
terrain like that, then the room is not checked. The BADROOM\_BITS
restriction can be overriden by the move\_flags a thing has. If it's
capable of flying or underwater breath, for example, then ROOM\_AIRY
or ROOM\_WATERY and ROOM\_UNDERWATER flags are allowed in tracking.

\end{enumerate}

After all that, if the thing is not here, but a path is found, then
the mob takes a step in that direction. A special case is if the mob
can cast spells or has a ranged weapon. Then, it will check first if
the thing it's hunting is nearby. If it is, then it will not move into
that room, but will range attack.

If this is the location we desire, or if the thing we desire is here,
then we do something based on the kind of hunting we were
doing. Possibilities include killing something, setting up a new
society, setting up a guard post, gathering raw materials, and
building a city. If you want mobs to do other things, this is where to
add special functions and hunting types, so when you send them
someplace, they will be able to do something.
