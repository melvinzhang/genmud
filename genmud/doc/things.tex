\chapter{Things}

\section{HOWTO}

If you want to add a piece of data to a thing, I strongly urge
you to stop for a moment. There are interchangable chunks of
data called values that are discussed in the next chapter that
will let you add data to the things. If you want to add data
to a player only, then add things to the player struct discussed
later on. If you are doing fundamental changes to the game,
then by all means change the thing struct, but it's not a good
idea to load it up with lots of special information.

If you insist upon doing this, you must recompile the entire game,
and you must make sure you take care of the creation and
destruction and copying of the thing if the piece of data you
add is a pointer (such as a string). The files are text, not
binary, so you will also have to take care of reading and writing
the new information to files.

\section{Introduction}

Things are the most important elements of the server. They are the
core of the database. When you do something within the game, you are
using the thing that represents you, to affect other things within the
game. If you are familiar with Diku type MUDs, then things in this
code take the place of areas, rooms, mobiles, and objects. You can use
aspects of more than one of those types (such as a mobile you can pick
up and use as a weapon, or enter and use as a boat (room)), but in
some cases the generalization breaks down. However, it does give a
great deal of flexibility, and I have found that I don't need to do a
lot of special purpose coding to get things to work.

\section{Nodes and Hierarchy}

One aspect of things is that they are {\it{nodes}} in a giant
tree. That means they have parents (th$->$in), they have children
(th$->$cont), and they can be a child within a thing
(th$->$next\_cont). The way in which things are kept organized is by
using a hierarchal structure.

The top level of this structure is the\_world. the\_world is a thing
which is not in anything, but is sort of the root of the rest of the
structure. the\_world has a linked list of contents (children), which
are the areas. In general, you cannot affect this relationship (and if
you can tell me so I can fix it so you can't), and it is important not
to mess with this relationship since it is required for a lot of the
game interactions that areas are the contents of the\_world.

The areas each have a bunch of children, as well. These children are
the prototypes for rooms, objects and mobiles. IT IS CRITICALLY
IMPORTANT THAT YOU MAKE SURE THAT THE ONLY THINGS INSIDE OF AREAS ARE
PROTOTYPES (OR ROOMS) AND THAT ALL PROTOTYPES END UP INSIDE OF AN
AREA...THE CODE DOES THIS NOW BUT THIS FACT IS USED IN MANY PLACES TO
DETERMINE WHETHER OR NOT A CERTAIN OBJECT IS A ``MASTER'' COPY
(PROTOTYPE) OR A ``CLONE'' WHICH IS OUT IN THE REAL WORLD SINCE MASTER
COPIES ARE IN AREAS AND THE CLONES WILL NOT BE IN AREAS!!!!

There are special commands to set up objects and mobiles, and rooms
are set up automatically based on their vnums relative to the area
vnum. Essentially, each area has a vnum (index number). Each area when
it is created also gets a number of ``things'' that can be in the
area. When the area is created, 3/4 of those available vnums are
considered to be ``rooms'', and 1/4 can be used for other things like
objs and mobs. The ``level'' of an area is how many total things it
has in it (in other words it blocks those vnums from being used within
another area), and its ``armor'' is the number of rooms available in
the area. These two numbers: level and armor can be changed if needed,
but be careful, especially if you decrease the numbers and you end up
with things inside the area itself which end up being outside the
appropriate vnums for the area.

 So, as an example, suppose an area has vnum = 40000, its level is 250
 and its armor is 150. That means this area uses vnums 40000-40249 and
 vnums 40001-40150 are rooms and the rest can be mobiles or
 objects. If you edit create $<$vnum$>$ and that vnum is between 40001
 and 40150, it will automatically be created as a room, and this
 cannot be changed later on. I know it sucks, but I was running into
 problems since rooms need to be special in many ways, so be sure you
 plan out your areas beforehand and give yourself extra space so you
 can have the areas grow later on if needed. Note that you cannot use
 the same number for more than one thing...so no more making mob 40003
 to sit in room 40003 popping object 40003. Sorry about that,
 too. Also, there can only be one copy of any rooms since I don't have
 exits set up to reference a vnum and a specific copy of that vnum. If
 you want things like multiroom ships and stuff like that, you should
 probably redo the code so that every thing gets a vnum and a copy
 number.


Finally, all of the action in the game takes place inside of the
rooms. You walk from room to room, fight stuff in there, drop things
in there, etc... You can also enter some objects (if they are set up
correctly), to make boats and things like that, but you cannot save
characters within those things because, again they are not as stable
as rooms, and it is possible to have 10 copies of boat umber 24323, so
how do you know which one you are in...since I didn't put in copy
numbers along with vnums. Again, if this bothers you, you should redo
the code adding copy numbers so you can really really have a totally
persistent world.

The hierarchy is also used for the core game functions. Since
everything is in something, can contain things, and can be in the same
thing as other things, we have these three layers of what a thing can
``see''. Generally you can interact with the thing you are in, the
things you contain, and things inside the same thing as you. When I
was thinking about the paradigm (yuckword - sorry) I was going to use
for the game, I decided the most important thing was this hierarchy,
and the important interactions were how things could move other things
within the hierarchy. Think of the most basic commands in the game:
enter, leave, e, s, w, n, u, d, drop, get, give, take, wear,
etc... These things all end up moving a thing from somewhere in the
hierarchy to some other place in the hierarchy. The main differences
between things are how they can move things, and can be moved within
the hierarchy. More on this below with Thing Flags.

\section{thing\_hash[]}

Another way in which the data is stored in a really big hash
table. If you are used to Diku code, the hash table

extern thing\_hash[HASH\_SIZE];

essentially takes the place of area\_list, obj\_list, mob\_list,
mob\_index\_hash[], obj\_index\_hash[], and room\_index\_hash[];. Since
everything has a vnum, it gets placed into the appropriate list. As of
this writing HASH\_SIZE is 57129. So, you wont probably have too many
prototypes within each hash table. This table is mainly used by THING
*find\_thing\_num(int num) which goes down the hash table looking for a
prototype with a vnum equal to num. When the thing editor is used to
create prototypes, and when copies are created and sent into the world,
they are added to the thing\_hash table in such a way that the prototypes
go first, and then the copies of things go after that. This is designed
to make find\_thing\_num work a little faster.  

As a quick note, PCs have a vnum equal to 0. That means they all lie
in thing\_hash[0], but I still use thing\_hash[PC\_VNUM \% HASH\_SIZE] in
most places where I rememberd to do so, just in case someone changes
it someday.

The hash table is doubly linked so that if you destroy a whole bunch
of things, you don't run into $O(N^2)$ problems deleting all of those
elements.

The hash table is used to update a lot of things in the world, and to find
say 57.squirrel if you have some reason to find it within the world. 
The find\_thing\_world function first tries to find the thing where the
initiator is, but then looks through the hash table if it can't find it.



\section{Thing Flags and Move Thing}

Ok, now what makes things different from each other? I figured out
that the idea that differentiates things from each other is how they
can move within the hierarchy, and how they can be moved within the
hierarchy. There are also a few other things like whether the thing
 can talk or fight, but those are more special cases. 

Since the database has a hierarchy, if we want to move something,
there are four things (actors) involved.

\begin{itemize}

\item The {\it{initiator}} which is the thing that originally gives the command to try to move something.

\item The {\it{mover}} which is the thing actually moving.

\item The {\it{start\_in}} which is where the mover was before something tried to move it.

\item the {\it{end\_in}} which is where the mover wants to end up when all of this is done.


\end{itemize}

So, for example in order for a thing to move, the initiator has to be able to move something (either itself or another thing depending on the situation), the mover has to allow itself to be moved by the initiator (which might be itself), and the start in has to allow the mover to leave itself, and the end\_in has to be willing to accept the mover.

There are many thing\_flags which cancel out some or all of these situattions, so if any of those conditions fail, you have a problem and you can't move the thing in that way. The thing\_flags in general BLOCK certain kinds of activities from taking place, so a thing with no thing flags will be able to do just about anything. I don't recommend this, however since that may lead to problems. 

Here is a list of the thing flags and what they essentially allow you do block certain commands:

\begin{itemize}

\item TH\_NO\_TAKE\_BY\_OTHER 0x00000001    -- initiator cannot move mover unless the initiator is the mover. 

\item TH\_NO\_MOVE\_SELF     0x00000002   -- No move self in or out, or ``walking'' around.

\item TH\_NO\_CONTAIN       0x00000004   -- NOTHING EVER IN THIS. You cannot put stuff in this or give something to it. Remove this bit from an object to make a container.

\item TH\_NO\_DRAG          0x00000008   -- Drag is special for things like corpses that cannot be picked up.

\item TH\_NO\_GIVEN         0x00000010   -- No take stuff from others - It will not let someone give something to it. 

\item TH\_NO\_DROP          0x00000020   -- Other no remove this from self -- It will not let you drop or give it to someone.

\item TH\_NO\_CAN\_GET       0x00000040   -- Cannot get anything itself -- This thing is not capable of picking something up and putting it within its inventory. 

\item TH\_NO\_MOVE\_BY\_OTHER 0x00000080   -- Others cannot move this but it can move itself (like mobs can walk, but you can't push them around (generally). 

\item TH\_NO\_PRY           0x00000100   -- Cannot be pried off or removed This is essentially noremove flag.

\item TH\_NO\_MOVE\_OTHER    0x00000200   -- This cannot move others -- Maybe it can move itself, but it cannot move anything else.

\item TH\_NO\_TAKE\_FROM     0x00000400   -- Keeps all inven...cannot be looted -- it just won't let you remove anything from it.

\item TH\_NO\_MOVE\_INTO     0x00000800   -- Other cant move self into this -- useful for normal objects. Disable this if you want a ``boat''.

\item TH\_NO\_LEAVE         0x00001000   -- Other cant move self out of this -- this is useful for rooms so there is no way to ``leave'' them. You can also use this for a type of ``cage'' or trap or something where someone goes into it and then can't leave.

\item TH\_NO\_REMOVE        0x00002000   -- This cannot be removed -- this is weaker than TH\_NO\_PRY, since there may be a ``pry'' command someday that lets people remove noremove eq.

\item TH\_UNSEEN           0x00004000   -- Cannot be seen unless you are an admin. This is super duper invis for things you really don't want players seeing.

\item TH\_NO\_FIGHT         0x00008000   -- This thing does not fight -- useful for objects.

\item TH\_NO\_TALK          0x00010000    -- This cannot communicate, it also in means that the thing is not intelligent.

\item TH\_MARKED           0x00020000   -- Mark this thing temporarily, used internally.

\item TH\_CHANGED          0x00040000   -- This thing has been changed.. useful
  for editing. It tells the areas when to save.

\item TH\_IS\_ROOM          0x00080000   -- This is marked as room. Cannot
  change this inside the game for reasons of safety and consistency.

\item TH\_IS\_AREA          0x00100000   -- This thing is an area. Cannot change  this inside the game for reasons of safety and consistency.

\item TH\_NUKE             0x00200000   -- This thing does not get saved into the area file.

\item TH\_NO\_PEEK          0x00400000   -- Things can't look into this to see its inventory. Useful since you can look into a chest in your inventory and see its contents, but you don't want people to be able to see into a player's inventory, or a mob's inventory.

\item TH\_SCRIPT           0x01000000   -- This thing has a script  attached to it. I do this just so that it's a little faster going down the event list for repeated events if you KNOW something isn't gonna have a script associated with it.

\item TH\_DROP\_DESTROY     0x02000000   -- This is destroyed if dropped. 

\item TH\_NO\_GIVE          0x04000000   -- You cannot give this thing, but I guess you can drop it.

\end{itemize}



\section{Starting to Edit}

To edit a thing, you use the editor. There is only one editor. That
means no medit, no oedit, no redit, and no aedit. There is only
``edit''. There are several ways to invoke the editor.

If you are inside a room, just type edit, and you will start to edit
that room. Also, if you are editing the thing you're inside, when you
move newsud, the thing you're editing follows you, so you're always
editing the thing you're inside. This doesn't work if you're not
editing the thing you're in.

You can also type edit $number$. In this case you edit the thing with
that number. If you want to create a thing, you can type edit create
$number$. If that thing already exists, then you just start editing
it. You can also type edit create by itself. If you do this, and you
are editing something already, then the next available number (inside
the area you are editing) will be created and you can start to edit
that. Note that when you create things, the game checks if your new
number is inside of an area already or not.

Finally, you can type edit create area $startnum$ $numthings$. Note
that the game does checks to make sure that every thing from startnum
to startnum + numthings is not already contained within an area (or it
should).

``edit'' also accesses the line editor for editing script code, but I
will come back to that later on.

Once you are in the editor, your commands will be interpreted as trying
to alter the thing you're editing. Every time you enter a command, you
will be shown a screen which lists all of the information for the thing.
The next section lists all of the various commands you can use while
in the editor.

\section{Editing}

This section lists all of the commands you can use in the editor.
Some pieces will be pushed back for explination in later chapters
since they are more complex. Also, the editing of exits will be dealt
with in the next chapter with the rest of the values.

The rest of this section will consist of a keyword you enter, and then the type of data that follows, and what it modifies.

\begin{itemize}

\item armor -- This lets you edit the armor of the thing. If you leave this at 0, then the armor is autocalculated when the thing is made. For areas, the armor is the total number of things in the area. Be careful changing this for areas, since there aren't any sanity checks for whether or not you're dropping things from the area.

\item cost -- sets the cost of the item. This is randomized on creation.

\item copyfrom -- This copies a description from the number specified.

\item copyto -- This copies the description of what you're editing to a new thing.

\item color -- this is a number from 0-15 which follows the current color scheme. Color doesn't really do anything now since I'm not showing mobjects in the overhead map.

\item desc or descr -- This puts you into the line editor and lets you edit the thing's description.

\item done -- this quits you out of the line editor.

\item flag or flags -- These will be dealt with in the chapter on flags.

\item hpmult or hp\_mult -- This sets the hp multiplier for a mobject. 10 is ``normal'', and the range can go from 2 to 100. Btw, the hps of the prototype are used to store this to save memory.

\item height -- this sets the height of the thing (in inches).

\item level -- this sets the level of the thing. This is useful for mobs since it determines their relative power. For objects and rooms this is good since it determines how well they cast spells (like potions and whatnot), and it also determines how resilient an item is to damage. The level of an area is the number of ``rooms'' in it. So, the remaining items are the armor - level. Be careful when setting this on an area...since you might end up screwing things up by either making rooms into nonrooms, or making nonrooms into rooms.

\item light -- This is the amount of light this thing gives off. You might set it on a room, or on a mob for kicks (and the room will be lit, and the mob will light up a room when it walks in...), but it is generally used on lanterns and stuff like that. 

\item long -- This sets the long description of the item. Unused in rooms, this is what you see when you type look and something is in the same thing as you. For areas, this is the repop message.

\item max or miw or max\_in\_world -- This lets you set the max in world for a thing. Areas and rooms have max in worlds equal to 1. Anything else can have a nonnegative max in world. If the max in world is 0, then the thing is unlimited. mv and max\_mv are used to store this to save memory. If you ever want to customize moves on things...altho I couldn't think of a reason why, so that's why I did this...then you will have to change some of the references in boot.c, edit, society.c, thing,c and thingio.c, as well as serv.h to fix it up.

\item name -- this sets the names string of the thing. This is the filename for an area. It cannot duplicate a current filename, and it must end in .are. It sets a special string you can use in rooms so you can goto it. Otherwise, it's a list of names. Be careful setting this on PC's since their filename is based on this word. 

\item reset -- resets will be dealt with later

\item size -- this sets the interior size of a thing. It determines the maximum amount of weight a thing can contain.

\item sex -- this is male, female or neuter. It sets the sex of the thing.

\item setup -- You either setup mob or setup obj. This sets the thing flags to the proper values for each of these base types. You can, of course come in later and change some of these. For example, a container is an object where it can contain things. A boat is a thing that lets people enter and leave it.

\item short -- This has several uses. In areas, this is the name of the area. In rooms, this is the name of the room you see. For mobs, this is the ``name'' you see when it does something..like ``a stocky dwarf''. For objects, this is the string you see when something is in your inventory, or worn by you or someone else. It is also the name you see when you pick something up: ``You pick up a long sword.'' Be careful when setting this on PC's since this is linked to the PC's filename.

\item type or typename -- This doesn't do much now, except in areas where this is the list of builders, and in other things when you look at them, it says that they are some type of ``orc'' or ``weapon'' instead of some type of ``thing''.

\item thing or things -- This lets you set a ``thing\_flag'' on the thing. This is segregated from all other flags because it is so important. Some thing flags like room or area flags cannot be set or unset. If you type thing nuke, this thing will not be saved to disk.


\item time or timer -- this sets the timer on a thing when it is created. For rooms, this does nothing. For areas, this is how long until the area resets. For other things, this is the number of hours it will exist after it's created and before it goes poof.


\item val or value or v -- this is for editing values which will come in the next chapter.

\item wear -- This sets the wear location of the thing. Every thing can only have one wear location to save space, but I guess you could change this to allow things to be worn all over if you feel like it.

\end{itemize}


The rest of the things you can edit will come up in the next chapters.


