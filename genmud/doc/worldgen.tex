\chapter{World Generation}

The game has extensive code for generating areas and other content because
I got sick of not having any builders and not being able to build myself.

\section{Worldgen Clearing}
The top level of the generation code is the worldgen code. It is accessed
by using the worldgen command. There are a few steps involved in generating
a world. First, you want to type {\bf{worldgen clear yes}} and clear all
of the old worldgen data so that you can make a new world. You can always
type worldgen clear yes to clear the generated world...but a word
of warning: If you do this you delete characters' eq so don't do this
once you get a world up an running! 


\section{Generate the Area Sectors}
Once the world is clear and you've rebooted, you need to type {\bf{worldgen
dx dy}}. dx and dy must be between 5 and WORLDGEN\_MAX -1 (currently 33). 
This will generate a grid of sector types that will be used to generate the
world. The general algorithm seeds some snow to the north, some 
deserts to the east, some swamps to the SW, some forests to the 
SE, and then some other fields and whatnot around. 

From these few seeds
the program repeatedly attempts to add new sectors next to existing
sectors by looking at blank sectors next to made sectors and
seeing if the sectors grow. The chances of each kind of sector 
being grown next to each other are given by the sector\_adjacency
matrix in worldgen.c. The types are the same from L-R and Top-Bottom.
The way it works is that the blank square gets all of the numbers for each
sector type added up from all adjacent squares, then a random number
is picked from that set, and the new sector gets made. This is an
asynchronous update. Within the function worldgen\_add\_sector,
look at the curr\_exits variable. Note that when it's large, it's
very unlikely that a room will be added in this room. This makes
the map stringier.

After every iteration of trying to add adjacent sectors, the map is 
checked for adjacency, and once all of the squares are touching, the
growth stops. This makes very stringy and curvy maps which IMO are 
interesting in that they give some nice chokepoints.

The levels of each area are also generated at this point by setting
the southernmost areas on the map (next to where the player outposts
will go) to level 10 or so. Then repeatedly, the map is scanned and
any level 0 areas that are adjacent to areas that aren't level 0 get
put to the previous area level + some constant. If there's more than
one area next to an area with level 0, it picks the S then N then 
average of the levels of the areas adjacent. The net effect of this
is that areas near where the players start are easy and areas far away
are hard.

\section{Worldgen Generate}

At this point, we can generate the whole world. The command to do this is
worldgen generate number. Number gives the approximate size that
the generated areas will be and if number is omitted, then it's
assumed to be 500. There is an upper limit on the number of things
that can be made, so if you're likely to use more than 400000 things
for the top world (say if you had a 20x20 grid of 1000 room areas)
then you need to increase the WORLDGEN\_VNUM\_SIZE variable in
worldgen.h.



\begin{itemize}

\item Societies get generated using the code in socigen.c. They get
mobs created and placed into area 112000 (currently) and they get
eq given to them that has a chance of repopping when the society
creatures are made.


\item Randpop mobs get generated. This uses the seeds in area 105200
(currently) to generate all kinds of simple animals and other kinds
of highlevel creatures like beasts and elementals that players can find
when they wander around the world. These are not meant to be really
complex things. They're just there to make up the background so that
players will see the expected mobs when they walk around the world.
Also note that these creatures have room sectors on them so that 
when they repop, they tend to repop in the "right" rooms.



\item Then the areas get generated. Each area is generated and its
starting vnum is placed into the worldgen\_areas grid and the
worldgen\_underworld grid. The level of each area is a random number
based on the level generated above when the sectors were created.
The underworld areas are stronger than the above ground areas in
the same location. The size of the area is also a random number based
on the size of the world requested. If no size is requested, it is
assumed that the area size desired is 500. Be aware that if you set
an area to size 500, probably only 1/2-2/3 of the area will be
rooms, and many of those will be things like trees and catwalks and
underpasses and other things that get added on later. The initial
rooms used in the area are generally 40pct of the size of the area.

\item Then the game links the areas. It links above ground to 
above ground and below ground to below ground, then each area has
a link from the overworld to the underworld. The rooms used to link
up are the dir\_edge rooms made in the areagen function. 


\item Then, catwalks are added between areas. When the trees and
catwalks are made in the areagen code, some trees are labelled as
dir\_tree trees which means they're the most north, south, east
or west in the area. Those trees on the extremes get linked to trees
in other areas (if possible). The catwalks here are actually two halves
that get spliced together so one catwalk isn't "hanging over" another
area quite so much.

\item After that, the societies get placed into the world. Since each 
society generally has some room sector flags associated with it,
the societies are placed into areas with some of the same
room sector flags that the society has. Each area can hold at most
one society at this point.

\item After that, the player outposts (areas 1000 and 1050) are 
linked to the rest of the world on the southern edge of the map. 
For that reason, make sure that you have a few areas on the
southern edge of the map.

\item Following the outpost addition, some simple quests (fedex quests)
are generated using the questgen code.

\item The history of the world including deities, powerful races, and the
  history itself are generated.

\item The overall map of the world is saved under the help WORLDMAP
  helpfile. That way players can have an idea of the shape of the world
  without knowing all of the details.

\item Then the world is purged, the teachers of various skills are
  created and sent into the world, and the overhead map symbols for the
rooms in the world are cleaned up.

\end{itemize}
