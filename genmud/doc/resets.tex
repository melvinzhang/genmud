\chapter{Resets}

\section{Introduction}

Everything is a thing as we have seen in the previous chapters. One
place that this becomes very useful is when dealing with resets.
A reset is essentially a periodic repopulation of the world with
things. This occurs a few times an hour per area, although the
specific reset times for each area are randomly staggered. 
(This can be changed in upd.c..look for reset\_thing in upd.c,
and work with the numbers in that general area of the code. The timer
on the thing counts down to when it gets reset.

Resets here are lots of fun. They also save you work if you use
them correctly. As a quick example, if you want 10 guards to reset
in a room with randomly popping armor, using the old methods,
you would go to the room and make 10 resets of a guard. Between each
guard reset, you would have to put reset for armor and weapons,
and it would end up being a big mess with 50-60 resets in the
room. Using this reset code, you put a reset of 2 random weapons on
the mob, a reset of 5 random armors on the mob, and then you put a
reset of 10 guard mobs in the room. Using recursiveness and randomness
you took a job requiring 50-60 resets and turned it into one requiring
3 resets.


\section{The Reset Structure}

The reset structure looks like this:

\begin{verbatim}

struct reset_data
{
  RESET *next;
  int vnum;
  unsigned char pct;
  unsigned char nest;
  unsigned int max_num;
};

\end{verbatim}

The next is there because these are stored in a list. Note that
specific instances of mobjects don't get the resets copied.
Only the prototype keeps that information, since there didn't
seem to be a need to add it, because you can just look at the
prototype. 


 The vnum is the
vnum of the thing being reset. However, in many cases, the vnum of the
reset is not the vnum of the thing being created. This is because of
random repops. You can see VAL\_RANDPOP in the chapter on values for
some information on this, although more will follow. When a thing
has a randpop value on it, that tells the reset code to reset another
thing instead of the randpop item.

The pct is a number from 0 to 100 that gives the percentage chance of
this reset being called when it's accessed.

The nest is how deep in the reset stack this item should go. As an
example of this. Suppose you have a room with a chest in it, and you
want to reset a blanket inside the chest. Then the chest would
be reset with a nest of 1, and the blanket would be reset with
a nest of 2. This is useful if you want to have the chest resetting
in several locations with different items in it at each location.
However, if you only had one instance of the chest resetting, you
could make a reset of the chest in the room at nest 1, and a reset
of the blanket on the chest thing with a nest of 1. This nesting
property is very powerful, and can do very bad things, so be careful.


The max\_num is the max number of this item that will pop in the thing
being reset. There are several cases of how this works.  A quick note:
the resets are totally recursive, so if you have say 100 mobs popping
in an area with 100 items on them each, you make approximately
10000 items. So, if you have 1000 mobs with 1000 items, you get 
1 million items. Be careful with this, because I don't have any
safeguards on this in case you have some reason to have such
massive numbers of resets. You have been warned, and you will
be warned again. If you get too many items, the game will shut down
since there's an upper limit on the number of things that can
be created to keep the game from eating up all the resources
on the computer.

If the thing being reset is the start area, the reset is a ``global'' 
reset, and the max number is capped by the max number of this
thing in the whole world.

If the thing being reset is another area, then the reset is an area
reset, and the max num is capped by the total number of things
of that type in the area. This is a useful distinction, since
if you wanted to reset a gray wolf all over the MUD, you could
either make a global reset, or you could make a single gray wolf mob
and in each area with a forest, you could reset 5, 10, 20 wolves
somewhere in each of those areas...and the game will check the
total population in the area you are resetting, and use that
number to determine how many more get popped.

If the thing being reset is anything else, from a room to a mob
to a chest or something, then the max pop is checked against the
number of things of that type already in the room or chest. This
is useful because if you want 10 gateguards at a gate, you just
make a reset in the room with a max\_pop of 10, and the gates
will reset with 10 guards. Also, if 5 guards get killed, then the
gates will reset back up to 10 guards during the next reset.

A note about random resets. There are random resets, which let you
pop items from certain ``tiers'' you can set up. If there are 
N tiers of items and K items per tier in the random repop, then
the max\_pop checks against all things with numbers from the
starting number of the random repop up to the starting number plus
N$\times$K. It will make more sense down below in the part on random
repops. 

\section{The Reset Command}

If you are editing something, then the word ``reset'' will try to
make a reset on that thing. In that case, use the command ``rese''
instead of ``reset'', or else the game will keep trying to put
resets onto the thing you're editing.

If you type ``reset'' by itself in a room, you will get a list
of all the resets in that room. 

If you type reset area, you reset the area you are currently
inside of. Note that if you're in a room in the start area, you
will do those ``global'' resets.


If you type reset room, you reset whatever room you're currently 
inside of.

\section{Editing resets}

When you're in the editor, you can edit resets. You will notice
that they look like a list. Type edit 1200 to see a list of a 
lot of resets. They look kind of like this:

\begin{verbatim}
Reset 1: (1613 - a small boy) (100 pct) (20 max) (1 nest)
Reset 2: (1614 - a small girl) (100 pct) (20 max) (1 nest)
\end{verbatim}

They are all ordered starting at 1. You can delete a reset by typing
reset $<$number$>$ delete. You can also add resets in at different
locations by typing the number of the reset you want to push down
the list. So, for example, if you wanted a reset before item 8,
you would start the reset command with reset 8 ....

The whole syntax for a reset is as follows:

\begin{verbatim}
reset <number> <vnum> <percent> <max_num> <nest>
\end{verbatim}

You don't have to enter all of the numbers. You can stop at the
vnum. If you do this, here are how the remaining numbers are set:
 the percent is assumed to be 100, the max\_num
is assumed to be 1, and the nest is assumed to be 1. 

A quick reminder. Resets are totally recursive, so if your max\_num's
are too big, you will get massive numbers of things being created.

However, this recursiveness is nice to have, since it makes your
life easier. Say, for example, you wanted some rangers patrolling
a forest. You wanted them to have backpacks filled with randomized
equipment. What you could do is make a reset of the rangers in
the area, then on the rangers you put a reset of a backpack, and
either in the ranger, or in the backpack, you put a reset of
some randomized equipment. Much easier than having 10 different
spots with a ranger repop, followed by a backpack repop, followed
by some random repops of stuff to put in the backpacks.

\section{Room ``Sectors'' and resets}

When using area/global resets, it is possible to use room sectors to
determine where things will get reset. The room sectors in this
code are room\_flags. One of the reasons for it was so I could
get this random reset code to work. If you have something being
reset as an area reset, and it has no roomflags on it, it gets
reset to a random room anywhere in the area. If it has certain
roomflags, like desert, or forest, but not nomagic or extramana,
then it will only get reset into rooms with at least one of the
flags the same. So, you could make a ranger that resets in forest
or field rooms, or a guard that resets in easymove (road) or inside
rooms in a city. This lets you have random repops of several
different kinds of creatures within one area, and you don't
have to worry about stupid stuff like fish on the highway, or 
people at the bottom of the ocean. 

\section{Random Resets}

So far, I have talked about how to make randomized resets in terms of
areawide or worldwide or percentagewise resets. Now, I will explain
the idea behind ``randpops''. As I said before, there is a ``value''
called VAL\_RANDPOP which is explained in the chapter on randpops,
but I will go into detail here.



My thinking is the following. In MUDs, there are generally several
``qualities'' or ``ranks'' of equipment that perform a single
function. There might also be several interchangable items (like
herbs) and you might want to load any one of a large number of
those items. You might also have several ranks of several kinds
of items, and you might want to decide to say take a random item
from rank 2 or something like that. This comes up with armor,
weapons, and gems...since there are several ranks of each kind
of item, but within each rank, there are several different items.
So, we will want to be able to pick a rank of item, and a specific
item within that rank. It would also be nice to be able to specify
say ``a gem reset'', and have the game determine the power of
the gem made based on the level of the mob with the reset.

This also comes up in other ways. Suppose you have about 15
different ``forests'' in your MUD. You need to populate each
of these forests with cruddy little creatures that people
just walk by and ignore. You can't just leave them out,
because people will complain that the areas are blank.
What you can do is the following. Make an area called
``forestmobs.are'' or whatever name you want to give it,
and inside of that area...you make 50? 100?...some large number
of forest creatures. You don't need 10 kinds of ``a gray squirrel''
in your MUD, although that's your choice if you want to.
You could even set them up in tiers, so that you have
small creatures...rabbits/squirrels...and somewhat larger
creatures...fox/wolf...and even bigger things like bears. Then,
you could use this randpop idea to make sure that on average
there are 5 little creatures per medium and 5 medium per large
or something like that. Or, you could just randomly pick from a 
list of forest creatures. Either way, once you're done all you have
to do to populate subsequent forests is put a reset in the area
saying you want 20..50..200? ``forest creatures'' reset in your
area, and it will all be taken care of for you. You could
even just have a global reset with a huge number of forest creatures,
and over time that would just populate the whole world,
but this becomes trickier as you will have to keep track of how 
dense the population is and up it as time goes on or else there
will be fewer and fewer creatures in more and more forests.




Within the VEL\_RANDPOP value, here are what the various numbers
represent:

\begin{enumerate}

\item val[0] = vnum where you start popping stuff.

\item val[1] = number of items per tier (rank).

\item val[2] = number of tiers of items

\item val[3] = Check vs level of the thing being reset onto. This 
	means if nr (1,val[3]) < level\_of\_thing, you advance a
	tier. This is useful for stuff like gems, so that a highlevel
	mob has a large chance of resetting a good gem,
	while a lowlevel mob has only a small chance of resetting one.
See item 277 for an example of this. The level in that case is 300,
so even a level 150 mob only has a 1/16 chance of resetting a super
gem...and that's after you consider the reset chance is something like
5 percent on average. However, a level 300 mob (pretty much unkillable) would
have a 100 percent chance of popping a super gem.

\item val[4]/val[5] These are used together. If you set val[4] larger
than 0, and val[5] larger than 0, then when you check to advance a
tier, you check if nr(1,val[4]) < val[5], you advance a tier. This is
useful for something like resetting minerals. If you wanted to have
a 1/3 chance of advancing to the next mineral, you would set
val[4] to 3, and val[5] to 2. Then you only advance if nr(1,3) <
2..which would only happen about 1/3 of the time. 
\end{enumerate}

So, how do you have to set up the items to make this work? There are
two basic ways. First, the items must be all in a row after the val[0]
number. Sorry, but coding something that was totally general would be
too hard and messy. So, the normal setup is to have N tiers, and K
items per tier. To set this up, you put the K items in the first tier
in a row, starting at val[0]. For examples of this, look at how the
standard weapons are set up starting at 600 (randpop item 270) and the
armors starting at item 635 (randpop item 271). Then, directly after
the K items in tier 1, you put the K items in tier 2. So, you end up
with a block of N$\times$K items in a row accessed by the randpop.

Ther is another way to do this, however. This is the method where
you put all of the items of the same type in a row, and then
make a sequence of those lists of items of the same type in 
increasing rank. An example of this is gems. Start looking at item
110, and note that the four ranks of diamond gems are all in a
row. Then, the emerald gems...etc. The way to get this to work with
the randpop code is to make the randpop word ``interlaced''. Perhaps
it should have been interleaved, but wtf I don't care :P. The idea is
that the items in the ranks are not sequential, but interweaved
between the verious ranks of a single item type. To see a randpop that
uses this method of setting up the items, look at item 277.

You can leave blanks in your setup...so if you have 10 crappy items, 8
good items, 6 very good items, and 2 awesome items...you can still use
this. It's just that your chances of getting the better items will be
decreased since when the code comes to a number where no thing exists,
it just ignores the command to reset it and moves on. 

Getting back to the previous section on max\_pop, this is where the
max\_pop checking gets affected by randpops. Instead of checking
the number of things present vs the vnum you are popping, it
first checks for a randpop, and if the thing you chose to reset
has a randpop on it, it checks for anything with a vnum in the
range val[0] to val[0] + (val[1]*val[2]). That way, you don't
get 5000 forest mobs in an area because there are 50 types of
mobs and a reset that loads 100 of them...so it won't try to
fill up the forest with 100 of each of those 50 types of mobs. It
will just fill up the forest with 100 things coming out of
the list of 50.

So what other kinds of setups can you do? You have seen the setups
with armor and weapons and gems with several ranks of items which have
several item types within each rank. 

You can also do things where you have one rank, and randomly pick an
item. The example I give for this might be resetting herbs in a forest
for people to pick up. You have 50 herbs, and just randomly pick
one. To do this, set val[2] to 1...so you have one tier. Then the code
will ignore val[3], val[4] and val[5], and just pick something from
tier 1. 

You can also do things where each tier only has one item in it, and
you pick from those tiers. This might work in the case of mining
minerals or something where you just have several ranks of minerals.

Some other stuff I've thought about would be like the following. Put
all of your superpowerful mobs in one place, and all of your
superpowerful equipment in another. Randomly reset the supermobs all
over the game (maybe 5-10 at a time), and put small chances of super
items on them, and randomly pick the super items. THAT would really
mess up the players who expect to be able to goto the same spot
and pop the same mob with the same eq each time. I haven't implemented
this, but it seems like a good idea...even if it would make players
mad since they will have to actually search all over. This would
work better if you had maybe 50 supermobs, but only several are 
reset globally at once, and if you have many super items. You would find
items coming into the game much slower...and ``most bestest'' kits
will be harder to come by. 

So, have fun with the resets, I think you will like them once you
see what they can do.
