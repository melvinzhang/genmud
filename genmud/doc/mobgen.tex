\chapter{Mob Generation}

\section{Random Animals Generation}

There is code to generate random animals so that you don't have
to create each one by hand. The way it works is that there are animal
prototypes in the MOBGEN\_PROTO\_AREA\_VNUM area (currently 105200)
and those get used to generate mobs within the MOBGEN\_LOAD\_AREA\_VNUM
(currently 101000). 

The way it works is that each thing in the proto area has a long
desc, a short desc and a name and a level and other information.
The code cycles through each word in the long desc, the short desc
and the name generating animals as it goes along. It only generates
10 animals for each name so that we can have say 5 long desc names and
8 short desc names, but we don't see every combination. It's to keep
the numbers under control, and to keep the patterns from being
quite so obvious. You can see that several kinds of animals are
grouped together like chipmunks and squirrels, as well as
groundhogs and badgers so that you don't have to come up with
different names for each similar kind of animal.

The protos are divided into two parts. Those which are below the
STRONG\_RANDPOP\_MOB\_LEVEL (currently 70) are declared to be
normal mobs that repop in any old area. The mobs above that level
are declared to be special mobs and are things like unicorns and
gargoyles and other "fantastic" creatures.

The game makes two passes and generates the weak mobs first, and
then the strong mobs second so that they can be in two tiers. Then,
once the mobs are set up, the numbers are checked and assuming that
the number of weak mobs is not much less than the number of strong
mobs, this will work. If the number of weak mobs is much less, then
the randpop won't work correctly and many of the strong mobs at
the end of the strong mobs list will not be loaded. This is due to the
nature of how the randpop code works. On the other hand, it shouldn't
be too hard to come up with a lot of simple background animal mobs.

The randpop item with vnum MOB\_RANDPOP\_VNUM (currently 278) is
altered to reflect the new mobs generated. Then, each area gets
a reset of this item with a certain number of tries associated
with it that will populate the area with some background mobs as
needed. 

The way the weak/strong dividing line comes in is that the chance
to advance a tier in the randpop (thereby making a strong mob
instead of a weak mob) is based on the area level, so lowlevel
areas will almost never have powerful mobs, and high level areas
will more often have powerful mobs.

These mobs do not load equipment, although some may be used in quests.


\section{Person Generation}


The person generation works a little bit differently. Each area has
many mobs made for it and those are set up to be as random as possible.

When a person is generated, she is usually given a name and usually
given a society name and is given a profession. The professions are
found in the area with vnum PERSONGEN\_AREA\_VNUM (currently 105000)
and it looks like a list of professions. 

When a mob is made, it gets set up and it gets the values and 
flags listed on the thing in the persongen area. VAL\_CAST mobs get 
their spells randomized.

The level of the mob is determined by the level of the area it's
being created for and the level of the persongen item. It's a sum of
the two with some randomization. That way you can find weak warriors
near homelands and OMFG super powerful warriors far far away in the
world. 

The persons generated then get some items generated on them that
sometimes have their names, sometimes have their society names,
and sometimes don't. These items are based on the level of the mob
but there's a lot of randomness here. Becauseof this, there will
be out-of-depth items that are powerful but pop on easy mobs.
It won't happen that often, but it's intentional since I think the
players like to have some "low hanging fruit" that they can get
easily. Too much "balancing" makes the game boring. :)


The persons also get some detection spells at higher levels and they
sometimes get set as area resets, and sometimes as room resets in
an individual location. They also get some stock resets like 
armor and weapons and people who can cast spells sometimes load
gems.

This code is also used sometimes in the areagen code to make other kinds
of mobs in hidden places.





