\chapter{Introduction}

\section{Quick Start}

If you are looking at this and are familiar with MUDs, here is a quick
start method for logging into the game to look around. You have
already untarred the files I hope, but if not here is how to do
it. You must be in Linux unless this has been ported.

You can untar the file in your main directory since it creates its own
subdirectory. Another thing to remember is that if you're reading
this, and this came as part of the tarfile, you can safely skip the
next few steps by default. :) Assuming the filename of the tarfile is
something.tar.gz, type gunzip something.tar.gz Then type tar -xvf
something.tar These commands will set up your mud into the proper
directories. type cd bfdmud then cd src and then make. These two
commands should make the executable, but if not, then you have some
debugging to do. If it works, type ./scs \& which runs the startup
script. You may also want to edit your crontab to make the game start
up on reboot.

Assuming your username is mudadmin, in your home directory type
crontab -e.  Then, type i and then type
\vv
@reboot cd /home/mudadmin/bfdmud/src ; ./scs \&
\vv

Now, hit the escape key and type :wq. That should make your mud boot
up when the server reboots.

Now inside the same shell, type telnet localhost 3300, or if you know
the IP/name of the server log into the server at port 3300.  Once you
are logged in, go through the process of making up a character and
once you are in the game, quit out. Assuming you are still in the src
directory, type cd ../plr. Now let's say your character name was
Fred. Pick an editor you like in linux (I use pico, emacs, vi
whatever..) and edit Fred. When you are editing Fred, you will notice
a series of keyword/data pairs. Go find the one that says ``level''
and make the number after it 110 (The maximum level). Save the pfile
and quit the editor and then telnet localhost 3300 again and when you
log in you will be max level. A few notes about ``imms''.

\begin{enumerate}

\item There are no backdoors which give you imm levels within the game
when you first log in...just seems like a bad idea since you should
disable it anyway when you make your first char.

\item Imms are not immortal. The only advantages an ``imm'' has are
that you have access to all skills if you want, you can go anywhere
you want if you choose, and you can see all if you choose. You can
also set your hps to massive amounts (typing sset fred hm 20000000
followed by restore fred will make you pretty much invulnerable).

\item You can die from permadeath as an imm...so be careful with
that. Also, there is *NO* slay command, but you can freeze or delete
players if you need to. Slay is out because of permadeath...so you
can't screw someone over in a fit of rage. I would recommend slay XOR
permadeath only.

\end{enumerate}

Now you have your admin character, and you have access to all commands in the game. Feel free to look around and do stuff while you read through this stuff.


\section{What is this?}

So, what is this thing? This is a MUD, or Multi-User
Dungeon/Dimension, Multiuser Undergraduate Destroyer, etc... Think of
a chat room. Think of message boards. Think of a community where you
can log on and meet your friends on the internet. Think of a game
where you can create an alter ego and improve it and collect stuff and
have fun over a long period of time. A MUD is all of these things. It
is a type of game/simulation which can be played by many people from
all over the world at the same time.

When you log on, you create a ``character'' which you will use for a
long time. This thing that you create will be your identity and a way
of keeping score, and what you will use to interact with the rest of
the world and the other players. There are many methods of
communication, many ways to act toward other people, and many ways for
other people to interact with you.


\section{History} 

So, what about my history? I started MUDding in Jan/Feb of 1991 on
eris.berkeley.edu 4000. I got so addicted it isn't even funny. I got
away from MUDs for a few years and then came back around '95/96 as the
net was growing in popularity as I stupidly decided to try a MUD out
``just once'' to see if they were still interesting. Needless to say,
I got hooked again, but the second time around the disease progressed
past stage 1: Player, into stage 2: builder, on to stage 3: admin, and
finally stage 4: write your own codebase because your current codebase
wasn't cutting it. So, I built and adminned and coded on an Emlen
server (which comes from NiMUD/TheIsles which comes from MERC, which
comes from Diku).

In the spring of 2000, I finally realized I could not do what I wanted
to do with the server, so I gave up on it and wrote
this. (Incidentally tracking and resetting objects on society mobs are
what finally did it.) (Someone also made this code available on
kyndig.com apparently under Diku-Merc-NiMUD(TheIsles)-EmlenMUD.)
The idea is to make something that the players
will think is similar to the Emlen code, but which has a lot more
stuff available in it. The builders and coders will realize that it's
done differently, so if you are used to Dikus, this may take a little
while to ``get'' the idea behind the code. Once you have the idea, I
think you will see that there are advantages so the architecture here.

\section{Server Information}

The basic idea behind the code is as follows: There is a piece of network code
which lets you connect using telnet, a database filled with things to explore
and interact with, and an autonomous world which is constantly doing things
even when you aren't there. There are also many commands players can use, and
there are many many many kinds of things that can be built for the world. One
thing to realize is this. The code for the network, the data, and all that
stuff is the \_minimal\_ setup that seemed to work. Technically I should have
a separate thread for the network code (which now exists), and technically I
should probably have a general database, but I was more interested in getting
a basic framework setup that worked, instead of making a tank that could
withstand anything and be scalable into thousands of users on many servers.

(I have also noticed that MANY people spend so much time working on
"the perfect programming framework for creating a MUD" that they
never get to doing anything interesting with the MUD. The framework
should just be a starting point. You need to use it to do something
interesting.)

\section{Things}

Things are at the heart of the database for the world. If you are
familiar with Diku code, then things take the place of rooms, objects,
mobiles, and areas. They are all the same struct (Note this is written
in C, not C++), and the way they are differentiated is by certain
``thing\_flags'' set on them. I figured out that one way to structure
MUDs was to consider everything in the whole world as part of a giant
hierachical database and that the differences between things came down
to how they could be moved around within this database.

For example, you can't go into areas, and can't really interact with
them. Rooms you can walk into, and they generally can't affect things
inside of them, and there really isn't an ``outside'' of them. Objects
generally cannot move other things, or themselves, and they can be
moved by other things. Finally, mobs can generally move other things,
but cannot be moved themselves.

These are not hard and fast distinctions, so you can make things like
boats which are objects that you can enter like rooms. You can make a
squirrel that works like a mob, but you can pick it up and use it as a
weapon. There are lots of different things possible.

Using a setup like this means much less repeated code, and generally
code that ``works'' when you need to extend an idea without you
needing to add much, if any, new code.

Also, much later after I started this, I began to notice that
there are many people in the MUD community who get mad when people
give away code and let "just anyone" start a MUD easily. If you're
one of those people, enjoy.

\section{Code Format}

A note about the code format. The indentation used is based on the way emacs does things: 

\begin{verbatim}

if (...)
  {
    ...
    ...
  }

while (...)
  {
    ...
    ...
  }

\end{verbatim}

instead of 

\begin{verbatim}

if (...) {
  ... 
  ...
  }

while (...) {
  ...
  ...
  }

\end{verbatim}


Also, there is no variable name prefixing. I find that it slows me down, and
so I didn't use it. However, here are some general guidelines for variable naming conventions:

Variables with the words buf or arg someplace in them are generally char arrays.

Variables like found\_something or found or is\_something are generally boolean.

Variables like i,j,k,l,m,n,count\_this, that\_num, etc...are generally ints.
Single letter loop iterators are only used in places where they will
only be used in simple loops not taking up more than a page of text.

Variables like r, s, t, u are generally char *'s used for parsing
strings. I have no idea why I started doing this, but all those for (t
= buf; *t; t++) things are cuz this is how I like to set up the
variables.

Variables like th, thn, thg, vict, victim, obj, objn, room, roomn,
area, arean are generally things. So, when you see ``th$->$'' all over
the place, that is just a pointer to a thing.

To find the code for a command you noticed in the game, like the
``foo'' command, type grep do\_foo *.c. That will give you the
location of the do\_foo function which you can then edit.

I also have pretty standardized functions for dealing with the
creation and destruction of data. Generally, when something is
created, it is created by use of a new\_foo(void); function, which
returns a pointer to the data. Also, note that if malloc fails, we get
a segfault and call it a day. I don't know how I would free up memory
on a system shared by lots of other MUDs at the same time, so I figure
it's safer to do it this way. When you want to destroy a piece of
data, you call the free\_foo(FOO *); function. No data gets destroyed.

My reason for doing this is that as the game goes on, you get little
peaks and valleys in terms of how much memory is currently in
use. Since all things generally keep using the same types of data:
things, pcs, values, flags etc...these things will be reused
eventually, so I figured it was better to keep the memory from
becoming too fragmented. I even try to avoid making/freeing strings as
the game goes along, (except where it seems necessary), to keep this
from getting too fragmented. The string allocation could probably
stand some improvement...maybe by making a giant block of memory set
aside for string allocation, but I haven't done that yet.

When memory is sent back to be reused, it gets put into a list which
is headed by foo\_free;, and a count of the number of data chunks of
that type is kept in the foo\_count variable. The memory command will
try to give you an idea of how much memory is being used.

Reading and writing to files are generally handled by read\_foo() and
write\_foo() functions. (I used to use load\_foo() and save\_foo() functions
but I took those out sometime ago for consistency.)

Also, the amount of documentation and comments in the code is spotty,
mainly because if the coding was easy I didn't bother with much
documentation, since I just went through and did the coding in one
pass without many changes or bug fixes. Places where the documentation
is thicker represent areas where I either had problems getting things
to work, or where I was constantly going back refining things. Some
good examples of this are resets, scripts and societies, each of which took a
long time to get working.

\section{Altering the Code}

This is a general rundown on how to do a few kinds of things
to alter the code.

\subsection{Adding Commands}

You will probably want to add commands. If you do this, you
might want to create a new file, or add the command to an existing
file. If you make a new file, you will need to copy some of the
header files to the top of the new file and then start typing
the function.

For the sake of consistency, try to name the command do\_XXXXX(). 
You will need to add the command to a header file (probably serv.h)
unless you're going to use this command in a major module. I tend
to put commands in serv.h anyway.

You will need to edit int.c and add an "add\_command" line, but
it's pretty simple to see what to do: The first argument is the
command name that gets added to the table, the second is the function,
the third is the min level to use the skill, and the fourth is
whether or not the command is automatically logged.

Just follow the format for the other commands: Bringing in args and
stripping off arguments with f\_word() to use to find objects nearby
and then set poiners to those objects (these vars should be set to
NULL first so you know if you never got to set it, it's not going to be
random data).

Then, do the calcations you need on the objects and modify them as
needed, and then send the outputs.

\subsection{Altering Data Structures}

If you want to alter a data structure of type FOO let's say, or if you
want to create a data structure of type FOO, there are several steps
to go through.

\begin{itemize}

\item If this is going to be a fairly big chunk of code, and if it's
not central to the code (like societies aren't central to the
code, and they're a big chunk of code), you're probably better
off making foo.h and foo.c to start with so that you can keep
things more separate. Assuming that you only update FOO's in a 
few places in the code, you won't have to \#include "foo.h"
in too many places.

\item You will either need to have a way to add FOO objects to
existing data structures, or a way to store them so that
they can be accessed as needed. This generally consists of a
static or global variable like foo\_list or foo\_table or a
FOO *foos; variable in another object that lets you put the
foos into the world somehow. For example, spells and 
societies are in lists, and flags, tracks, and values are
parts of THINGs.

\item You will need a new\_foo() function that returns a pointer
of type FOO and a "cleared" data structure. You will need to alter
this function if you alter an existing structure. One thing to do
is if you're constantly creating/destroying FOO's make a variable
called FOO *foo\_free that points to a list of FOO's you can use
without having to reallocate them constantly.

\item You will need a free\_foo() function that takes a pointer
to a FOO object and then either free's it or (more likely) clears
the data in the object, and puts it into the foo\_free list.

It is important that if any other objects are pointing to this
object, you find all of those objects and clear the data on
them, too. If you don't do this you will run into big problems.

This isn't always feasible, so there are other ways to deal with
this. The reason free\_thing() is so complicated is that things
are very tightly knit into the world with the possibility of
many things having pointers to a thing. Since it isn't easy to
check all things every time a thing is freed, when a thing
is freed, it gets done in 2 stages. First, the thing is removed
from the world and all things interacting with it locally (in
the same room) stop interacting with it.

There are other things (like tracks in rooms all over the world)
that are more of a pain to deal with. It wouldn't be fun having
a thing keep track of all the tracks it's left. It's doable,
but expensive. Therefore, when a thing is freed, it gets put
into limbo for a minute until all rooms are updated and during
the room update, the room removes all tracks of things that are
not actually in the world. The delay also lets the world saving
thread finish writing a string in the name of the thing 
before it attempt to free the string. 

So just be careful with free\_foo() if you have a complex object.
I hope you will never have an object as complex or as tightly bound
to the rest of the data as THING's are, so this probably won't
be such a huge issue.

\item You will need to create read\_foos() and read\_foo() functions
or alter them if they already exist. I use read\_foos() to represent
reading the whole file in, and read\_foo() is for an individual
member.

\item You will need to create write\_foos() and write\_foo() functions
or alter them if they already exist. They work the same as above.

\item You may need a copy\_foo() function.

\item You may need an update\_foo() function. If the FOO's are in 
some global list, you may need to make an EVENT to update it
every once in awhile, or if FOO's are part of something else,
you will need to update FOO's as a part of that update (but
you might want to have the FOO's have their own update so you
can remove it if needed more easily later on.

\item If FOO is part of a larger structure, it might be a good idea
to have add\_foo() and remove\_foo() or foo\_to() and foo\_from() 
functions so that the larger structure isn't playing with the
FOO's directly. 

\item You may need to create an editor for foos. The way I do
editors is that I have a do\_editfoo() function (roughly speaking)
that lets you start to edit an object. That sets your editing
variable in your PC data structure to the pointer to the object
you want to edit. It also sets your "connected" state to some
number (See CON\_EDITING and so forth for this.) that strips
off the input you enter and sends it directly to the editor.

If you do this, you will need to have a way to edit existing
FOO's either by name or number, and  you will need a way to
create new FOO's to edit. My preferred way to do this last part is
to let you either do fooedit create with no argument to create
the "next" foo like in spells and societies, and possibly to let you
specify a number like sedit create 134 to create spell 134.


The editor can work like the other editors where you strip off
an argument to tell what part of the FOO you're editing, then
going into a giant switch or if/else chain to get to get to the
code where you actually set the data.

\end{itemize}

There's a lot to think about here when adding data structures, 
but that's because it's complex. However, getting these ideas
will help you in whatever kind of coding you do, and the
functions I've mentioned above are the kinds of things you have
to worry about in general coding situations, so it's good to
test things out and learn how they work.
