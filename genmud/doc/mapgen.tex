\chapter{Map Generation}

\section{Introduction}

The map generation code generates what I consider to be interesting
maps by taking the basic Bresenham rasterization algorithm and spicing
it up. You can create straight lines or paths that go off at an angle
and even big rectangular blocks of rooms with the code, but it really
shines when you make complicated blocks of rooms.

\section{Use}

The way to use the mapgen function is to type
\vv
mapgen dx dy length width fuzziness holeyness curviness extra\_lines
\vv
The dx and the dy give the initial vector that the algorithm uses
for determining where it will go. At least one of these must be
nonzero. 

The length is how many rooms the generation goes in the direction of the
larger of dx and dy.

The width is how wide the path is (approximately) and it starts at the width,
unless there is fuzziness, in which case this number is randomized.

The fuzziness is a measure of how much the width of the line varies from block
to block. So, if the fuzziness is 0, the width of the block of rooms will be
uniform.

The holeyness is a measure of how many rooms are deleted from the block of
rooms being created. This feature is how it's possible to make complicated 
labyrinthes. A holeyness around 8 or 9 or sometimes 10 tends to generate
interesting maps. Less than that and the map is too full, and more
than that and the map is too empty.

The curviness is a measure of how much the central line deviates from its
assigned course. The line deviates if nr(1,thisnumber)<50. It is interesting
if it is around 15 or so.

The extra lines are little lines that are placed in various places
throughout the map before it's finished. I wouldn't recommend too many
of these.


Having said all of that, try out mapgen 1 0 20 10 3 8 15 1 and alias it to a
single letter like v or z and repeatedly make maps and you will see a lot of
interesting and extremely varied maps.


\section{Mapgen Combine}

The mapgen combine function takes 2 maps wth certain offsets and attempts
to OR them together into a single map. If it doesn't work, it bails out and
doesn't free the two mapgens entered into it. If it succeeds, only the
new map is kept and the two maps that were originally used are freed
for future use.



